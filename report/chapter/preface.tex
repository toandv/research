\chapter*{Mở đầu}

Trong lĩnh vực công nghệ phần mềm hiện nay đang xuất hiện rất nhiều các xu hướng,
phương pháp phát triển phần mềm khác nhau như điện toán đám mây, di động, big data 
và phương pháp phát triển phần mềm linh hoạt agile hay test-driven development...
Cùng với sự phát triển của hạ tầng mạng, các phần mềm ứng dụng ngày càng được xây dựng
và triển khai trên nền web nhiều hơn, thay vì được triển khai cục bộ trên các máy tính.

Để hỗ trợ việc xây dựng các ứng dụng web trở lên dễ dàng, nhanh chóng và hiệu quả hơn, rất nhiều
công cụ đã được tạo ra, đó là các web framework.Trên thực tế, hiện nay có rất nhiều các web framework, được xây dựng dựa trên các ngôn ngữ khác nhau, một số ví dụ điển hình như Java với các framework như Spring, Struts, PHP có Zend framework, Yii, Python có Django... Các framework được tạo ra phù hợp với một số loại ứng dụng nhất định, mỗi framework đều có các tư tưởng, triết lý riêng và các điểm mạnh điểm yếu, không có cái nào được cho là viên đạn bạc. 

Với tư duy của người phát triển, với mỗi miền ứng dụng nhất định thì nên cân nhắc và lựa chọn phù hợp, ví dụ : để xây dựng một hệ thống lớn, có nghiệp vụ phức tạp, đòi hỏi có kiến trúc tốt thì sử dụng Java và các framework hỗ trợ được cho là một giải pháp tốt, nhưng để xây dựng một trang web thông thường, ít mang tính thương mại thì sử dụng một framework nhỏ gọn là một giải pháp hợp lý hơn.

{\bf Ruby on Rails} là một web framework được viết bằng ngôn ngữ lập trình Ruby, là một framework có rất nhiều các ưu điểm như thời gian phát triển nhanh, chi phí thấp, hướng tới người phát triển, áp dụng các tư tưởng của phương pháp agile và test-driven development. Trên thực tế Ruby on Rails đã được sử dụng thành công trong các ứng dụng như github hay twitter. 

Dưới sự hướng dẫn của thầy {\bf TS.Trương Anh Hoàng}, tài liệu này sẽ hướng tới việc tìm hiểu về framework Ruby on Rails cũng như cách sử dụng để xây dựng ứng dụng.\newline
Tài liệu gồm các phần sau:\newline
{\bf Chương 1} : Giới thiệu về framework Ruby on Rails\newline
{\bf Chương 2} : Giới thiệu về ngôn ngữ lập trình Ruby\newline
{\bf Chương 3} : Kiến trúc của Ruby on Rails\newline
{\bf Chương 4} : Các thành phần của Ruby on Rails\newline
Và một số phụ lục đính kèm.
